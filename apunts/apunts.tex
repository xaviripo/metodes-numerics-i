\documentclass[a4paper,12pt]{article}

% Imports
\usepackage[catalan]{babel}
\usepackage[utf8]{inputenc}
\usepackage[T1]{fontenc}
\usepackage{amsmath} % mathbf and mathbb
\usepackage{amsfonts} % font stuff
\usepackage{amstext}
\usepackage{amsthm} % style for \theorem and co
\usepackage{lmodern} % font with all characters
\usepackage{fancyhdr} % header and footer
\usepackage{amssymb} % \nexists and other symbols

% Header
\pagestyle{fancy}
\fancyhf{}
\rhead{Xavier Ripoll Echeveste}
\lhead{Mètodes Numèrics I, 2n parcial}
\rfoot{\thepage}

% Sections
\theoremstyle{definition}
\newtheorem{definition}{Definició}
\newtheorem*{notation}{Notació}

\theoremstyle{plain}
\newtheorem{theorem}{Teorema}[section]
\newtheorem{corollary}{Corol·lari}[theorem]
\newtheorem{lemma}[theorem]{Lema}

\theoremstyle{remark}
\newtheorem*{remark}{Nota}

\begin{document}

\section{Interpolació}

\begin{definition}[Interpolació]
Diem que una funció $f$ interpola una altra funció $g$ en els punts
$x_0,\allowbreak\dots,\allowbreak x_n$ del domini de $f$ i de $g$ si
$f(x_i) = g(x_i),\allowbreak 0\leq i\leq n$.
\end{definition}

\begin{theorem}[Existència i unicitat del polinomi interpolador]
Siguin $(x_0, f(x_0)),\allowbreak\dots,\allowbreak(x_n, f(x_n))$ parells de
nombres reals arbitraris amb $x_i \neq x_j$ si $i \neq j$. Llavors, existeix un
únic polinomi $p$ de grau menor o igual a $n$ tal que
$p(x_i) = f(x_i),\allowbreak 0\leq i\leq n$.
\end{theorem}

\begin{proof}
Per inducció sobre $n$.

El cas $n=0$ és trivial: prenem el polinomi constant $p(x)=f(x_0)$.

Per $n=1$ tenim:
\begin{equation*}
p(x)=f(x_0)+\frac{f(x_1)-f(x_0)}{x_1-x_0}(x-x_0)
\end{equation*}

Suposem ara que existeix un polinomi $p_{n-1}$ que interpola $f$ en els punts
$x_0,\allowbreak\dots,\allowbreak x_{n-1}$. Llavors, imposem que:
\begin{equation*}
p_n(x) = p_{n-1}(x) + \frac{f(x_n)-p_{n-1}(x)}{(x_n-x_0)\cdots(x_n-x_{n-1})}
(x-x_0)\cdots(x-x_{n-1})
\end{equation*}

Comprovem que $p_n$ interpoli tots els punts
$x_0,\allowbreak\dots,\allowbreak x_n$. Per a $x_i, 0\leq i< n$, es té que
l'últim sumand es cancel·la i tan sols queda $p_{n-1}(x_i)$, que per hipòtesi
d'inducció interpola $f$.

En canvi, per a $x_n$ tenim:
\begin{equation*}
\begin{split}
p_n(x_n) &= p_{n-1}(x_n) + \frac{f(x_n)-p_{n-1}(x_n)}
{(x_n-x_0)\cdots(x_n-x_{n-1})}
(x_n-x_0)\cdots(x_n-x_{n-1}) =\\\\
&= p_{n-1}(x_n) + f(x_n) - p_{n-1}(x_n) = f(x_n)
\end{split}
\end{equation*}

Per demostrar la unicitat, suposem que existeixen dos polinomis interpoladors
diferents $p$ i $q$ amb aquestes propietats. Sigui $r(x)=p(x)-q(x)$.
Cada $x_i, 0\leq i\leq n$ és una arrel de $r$, però $r$ és de grau menor o
igual a $n$ i té $n+1$ arrels; pel teorema fonamental de l'àlgebra, $r=0$ i
per tant $p=q$.
\end{proof}\newpage

\begin{notation}
Siguin $x_0,\allowbreak\dots,\allowbreak x_n\in\mathbb R$, llavors
$\langle x_0,\allowbreak\dots,\allowbreak x_n\rangle$ és l'interval tancat més
petit que conté $x_0,\allowbreak\dots,\allowbreak x_n$.
\end{notation}

\begin{theorem}[Error en la interpolació]
Sigui $f$ amb $n+1$ derivades contínues en
$\langle x_0,\allowbreak\dots,\allowbreak x_n\rangle$. Si $p$ és un polinomi
de gran menor o igual a $n$ que interpola a $f$ en
$x_0,\allowbreak\dots,\allowbreak x_n$, aleshores:
\begin{equation*}
f(x)-p(x)=\frac{f^{(n+1)}(\xi(x))}{(n+1)!}(x-x_0)\cdots(x-x_{n-1})
\end{equation*}

Per a algun $\xi(x)\in\langle x_0,\allowbreak\dots,\allowbreak x_n\rangle$.

\end{theorem}

\begin{proof}
Fixem $x\in\langle x_0,\allowbreak\dots,\allowbreak x_n\rangle\setminus
\{x_0,\allowbreak\dots,\allowbreak x_n\}$ arbitrària. Definim:
\begin{equation}
G(x) = f(x) - p(x) - R
\end{equation}
\end{proof}

\end{document}
